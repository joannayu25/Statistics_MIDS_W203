\documentclass[fleqn]{article}
\usepackage[margin=0.75in]{geometry}
\usepackage{amsmath}
\title{W203 Tuesday 4pm Fall 2018 HW3}
\author{Joanna Yu}
\begin{document}

\maketitle
1a) What is the probability that the next customer will request regular gas and fill the tank? \\
\begin{equation} \begin{aligned}
P(R \cap F)  &=  P(F|R) * P(R) \\
			  &= 0.30 * 0.40 \\
			  &= 0.12\\	\\
\end{aligned} \end{equation}

1b) What is the probability that the next customer will fill the tank?\\
\begin{equation} \begin{aligned}
P(F) &= P(R \cap F) + P(M \cap F) + P (P \cap F)\\
		&= P(F|R) * P(R) + P(F|M) * P(M) + P(F|P) * P(P)\\
		&=0.3*0.4 + 0.6*0.35 + 0.5*0.25\\
		&=0.455\\ \\
\end{aligned} \end{equation}

1c) Given that the next customer fills the tank, what is the conditional probability that they use regular gas?\\
\begin{equation} \begin{aligned}
P(R|F) &= {P(R \cap F) \over P(F)}\\
		&= 0.12/0.455\\
		&=0.2637\\ \\ 
\end{aligned} \end{equation}

2a) Draw an area diagram to represent these events\\ \\ \\ \\ \\ \\ \\ \\ \\ \\ \\ \\  \\ \\

2b) What is the probability of getting a red, waterproof, cool toy?\\
\begin{equation} \begin{aligned}
P(R \cap W \cap C) &= P(R \cup W \cup C) - P(R) - P(W) - P(C) + P(R \cap W) + P(R \cap C) + P(W \cap C)\\
		&= (1-{1\over6}) - {1\over2} -{1\over2} -{1\over3} + {1\over4} +{1\over6} + {1\over6} \\
		&={1\over12}\\ \\
\end{aligned} \end{equation}

2c) You pull out a toy at random and you observe only the color, noting that it is red. Conditional on just this information, what is the probability that the toy is not cool?\\
\begin{equation} \begin{aligned}
P(\bar C | R) &= 1-P(C|R)\\
		&= 1 - {P(C \cap R)\over P(R)}\\
		&= 1 - {1/6 \over 1/2}\\
		&={2\over3}\\ \\
\end{aligned} \end{equation}

2d) Given that a randomly selected toy is red or waterproof, what is the probability that it is cool? \\
\begin{equation} \begin{aligned}
P(C | R \cup W) &=  {P(C \cap R \cup W) \over P(R \cup W)} \\
		&=  {P(C \cap R) + P(C \cap W) - P(C\cap R\cap W) \over P(R) + P(W) - P(R \cap W)}\\
		&=  {1/6 + 1/6 -1/12 \over 1/2 + 1/2 -1/4}\\
		&= 1/3 \\
\end{aligned} \end{equation}

3a) What are the maximum and minimum possible values for $P(A \cap B)$?\\
For P(A) = 1/2, P(B) = 2/3, the maximum occurs when A is a subset of B. The minimum occurs when there is minimum overlap between A and B.\\
Maximum = 1/2 \\
Minimum = 1/2 + 2/3 - 1 = 1/6\\ \\

3b) What are the maximum and mimim possible values for $P(A|B)$?\\
\begin{equation} \begin{aligned}
P(A |B) = {P(A \cap B) \over P(B)}
\end{aligned} \end{equation}
The maximum occurs when $P(A \cap B)$ is the maximum. Likewise for the minimum. \\
\begin{equation} \begin{aligned}
Maximum = {1/2 \over 2/3} = 3/4\\
Minimum = {1/6 \over 2/3} = 1/4\\ \\
\end{aligned} \end{equation}

4) Given that a Berkeley student likes statistics, what is the probability that they have completed w203?\\
Let C be the event that a student has completed W203, L be the event that the student likes Statistics.\\
$P(L|C) = 3/4 \\
P(L|\bar C) = 1/4 \\
P(C) = 0.01 \\$
\begin{equation} \begin{aligned}
P(C|L) &= {P(L | C)*P(C) \over P(L)} \\
	&= {P(L | C)*P(C) \over P(L|C) * P(C) + P(L| \bar C) * P(\bar C)} \\
	&= {3/4 * 0.01 \over 3/4*0.01 + 1/4*0.99} \\
	&= 0.0294\\
\end{aligned} \end{equation}
\end{document}